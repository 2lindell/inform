% Start with the standard cover sheet:

[[Cover Sheet]]

% and now add:

\noindent{\it
Inform is a natural-language design system for interactive fiction, first
created in 1993. To most users it seems a single unified tool, but in fact
is made up of core software, common to all platforms, combined with
substantial user interfaces written independently for Mac OS X, Windows and
Linux, and with documentation and examples. The core material is in turn
divided into:
\medskip\par\noindent
Chapters 1 to 14: the source code to the NI compiler, written in C
\medskip\par\noindent
Appendix A: the Standard Rules, written in Inform 7
\medskip\par\noindent
Appendix B: the template layer, written in Inform 6
\medskip\par\noindent
Each of these chapter blocks is divided up into one or more named sections,
which have both full names (``Grammar Lines'') and abbreviations (|12/gl|,
the 12 signifying Chapter 12). Finally, each section is divided into
numbered ``paragraphs'', some named and others not. Code can thus be
approximately located by ``postal codes'' such as |12/gl|.$\S$7.
\medskip\par\noindent
Of its nature, Inform must perform a computational task which is difficult
to specify formally, particularly since part of its aim is to cope well
with incorrect input from an inexperienced user. It has become a complex
program some 120,000 lines in length, and like all such it must mitigate
its complexity using internal stylistic conventions and principles of
organisation. To this end it follows the ``literate programming'' dogma of
Donald Knuth, an idea which had in any case influenced Inform's own design.
In LP, a single source (``web'') is both ``tangled'' into a functional form
and also ``woven'' into a typeset form suitable for human readers. Inform
uses its own LP tool, |inweb|, an adaptation of Knuth's |CWEB| which scales
better to large multi-target projects.
\medskip\par\noindent
The Inform project's main goal is to publish the entire core Inform code
base, beginning in April 2008 with public drafts of Appendices A and B,
approximately 600pp of material. These use only the simplest form of LP
where tangling is minimal, and the reader needs no previous experience of
the genre.
}
